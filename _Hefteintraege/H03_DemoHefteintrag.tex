%%%%%%%%%%
% \Hefteintrag[Optional: Bearbeitungszeit in Minuten]{Faktor Zeilenabstand}{Titel}{Inhalt}
%%%%%%%%%%
\Hefteintrag[15]{2.5}{Hefteinträge: Zeilenabstand und Arbeitszeitentimer}
{
    \ifbeamer\else\vspace{-0.6cm}\fi
    
    Ist einem der Abstand zwischen Überschrift und Text zu groß (kommt manchmal bei großen Zeilenabständen vor, aber nur im Druckmodus, da in der Präsentation der Zeilenabstand immer einfach ist), kann man vspace mit negativer Länge nutzen, um den Text hochzuschieben.

    Dieser Hefteintrag hat zweieinhalb-fachen Zeilenabstand, damit man besser die Lösung Lücke ausfüllen kann. Zum Ausfüllen eignet sich ein Faktor zwischen 1.5 und 2.0 am besten. Hefteinträge können auch eine Arbeitszeit haben, falls man das mal braucht. Funktioniert genauso wie bei Aufgaben. Dieser Hefteintrag hat 15 Minuten Arbeitszeit
}