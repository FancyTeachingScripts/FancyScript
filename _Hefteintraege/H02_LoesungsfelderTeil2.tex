%%%%%%%%%%
% \Hefteintrag[Optional: Bearbeitungszeit in Minuten]{Faktor Zeilenabstand}{Titel}{Inhalt}
%%%%%%%%%%
\Hefteintrag{1}{Lösungfelder (Teil 1)}
{
    \vspace{0.5cm}
    \emphColA{\textbackslash{}LoesungTikz\{Tikz-Diagramm-Code\}}: 
    % \LoesungTikz{Tikz Diagramm-Code}
    \LoesungTikz{
        % 1. Knoten definieren
        % \node [stil] (interner_name) {Angezeigter Text};
        \node [oval, text width=2cm] (mitte) {Verarbeitung};
        \node [oval, text width=1cm, right=0.5cm of mitte] (ende) {Ende};

        % 2. Knoten mit Pfeilen verbinden
        % \draw [stil] (start_knoten) -- (ziel_knoten);
        \draw [-{Stealth[length=3mm]}] (mitte) -> (ende);
        
        % Ein zusätzlicher, gebogener Pfeil mit Beschriftung
        \draw [{Stealth[length=5mm]}-, red] (mitte.south) to[bend right=30] node[below, midway] {Rückkopplung} (ende.south);
    }


    \emphColA{\textbackslash{}LoesungKaroTikz\{Tikz-Diagramm-Code\}\{Anzahl Kästchen\}}: 
    % \LoesungKaroTikz{Tikz Diagramm-Code}{Anzahl Kästchen}
    \LoesungKaroTikz{
        % 1. Knoten definieren
        % \node [stil] (interner_name) {Angezeigter Text};
        \node (mitte) {Kombi aus LoesungKaro und LoesungTikz};
    }{2}

    \emphColA{\textbackslash{}LoesungLine\{Lösungstext\}\{Anzahl Zeilen\}}: 
    % \LoesungLine{Lösungstext}{Anzahl Zeilen}
    \LoesungLine{Erzeugt in der Schülerversion die festgelegte Anzahl an Zeilen zum Schreiben in Seitenbreite.}{1}

    % \LoesungMulti{Lösungstext}{Anzahl Zeilen}
    \emphColA{\textbackslash{}LoesungMulti\{Falsche mit \textbackslash{}multi\{\} und Richtige mit \textbackslash{}Lmulti\{\}\}}: 
    \LoesungMulti{
        \multi{Eine falsche Antwort}
        \Lmulti{Die richtige Antwort}
        \Lmulti{Die andere richtige Antwort}
        \multi{Noch eine falsche Antwort}
    }
}