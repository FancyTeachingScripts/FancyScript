%%%%%%%%%%
% \Hefteintrag[Optional: Bearbeitungszeit in Minuten]{Faktor Zeilenabstand}{Titel}{Inhalt}
%%%%%%%%%%
\Hefteintrag{1}{Lösungfelder (Teil 1)}
{
    \vspace{0.5cm}
    Manchmal gibt es beim Abstand zwischen Überschrift und Inhalt komische Bugs. Dann kann man einfach am Anfang ein vspace einfügen und es korrigieren.

    Es gibt verschiedene Möglichkeiten, Lösungsfelder einzufügen. Im Präsentationsmodus werden sie in der Reihenfolge ihres Auftretens im Code nacheinander eingeblendet. Zusätzlich kann man mit \textbackslash{}pause weitere Stellen erzeugen, an denen eine Folie weitergeblättert werden muss, um den darauffolgenden Inhalt anzuzeigen (funktioniert auch in tikz Diagrammen). 

    Folgende Lösungsbefehle stehen zur Verfügung:

    \emphColA{\textbackslash{}Loesung\{Lösungstext\}}: 
    % \Loesung{Lösungstext}
    \Loesung{Kein Platzhalter in Schülerversion}

    \emphColA{\textbackslash{}LoesungReplace\{Lösungstext\}\{Originaltext\}}: 
    % \LoesungReplace{Lösungstext}{Originaltext}
    \LoesungReplace{Ich werde in der Lösungsversion angezeigt}{Ich werde in der Schülerversion angezeigt}

    \emphColA{\textbackslash{}LoesungLeer\{Lösungstext\}\{Höhe des freien Bereichs\}}: 
    % \LoesungLeer{Lösungstext}{Höhe freier Bereich}
    \LoesungLeer{LoesungLeer fügt in der Schülerversion einen \textbackslash{}vspace mit der angegebenen Höhe ein}{0.5cm}

    \emphColA{\textbackslash{}LoesungImg\{Lösungsbild\}\{Originalbild\}\{Breite\}}: 
    % \LoesungImg{Lösungsbild}{Originalbild}{Breite}
    \LoesungImg{sty/img/finger_black.pdf}{sty/img/finger_black.pdf}{0.03\textwidth}

    \emphColA{\textbackslash{}LoesungLuecke\{Lösungstext\}\{Breite der Lücke\}}: 
    % \LoesungLuecke{Lösungstext}{Breite der Lücke}
    \LoesungLuecke{Erzeugt in der Schülerversion eine Lücke zum Ausfüllen.}{6cm}


    \emphColA{\textbackslash{}LoesungKaro[Optional: Größe der Kästchen]\{Lösungstext\}\{Originaltext\}}: 
    % \LoesungKaro[Optional: Größe der Kästchen]{Lösungstext}{Anzahl}
    \LoesungKaro{Erzeugt Karo-Feld mit Seitenbreite und angegebener Kästchenanzahl in der Höhe. Optional kann Größe der Quadrate angegeben werden, ansonsten automatisch 0.5cm.}{2}




    \emphColA{\textbackslash{}LoesungTikz\{Lösungstext\}\{Originaltext\}}: 
    % \LoesungTikz{Tikz Diagramm-Code}
    \LoesungTikz{
        % 1. Knoten definieren
        % \node [stil] (interner_name) {Angezeigter Text};
        \node [oval, text width=2cm] (mitte) {Verarbeitung};
        \node [oval, text width=1cm, right=0.5cm of mitte] (ende) {Ende};

        % 2. Knoten mit Pfeilen verbinden
        % \draw [stil] (start_knoten) -- (ziel_knoten);
        \draw [-{Stealth[length=3mm]}] (mitte) -> (ende);
        
        % Ein zusätzlicher, gebogener Pfeil mit Beschriftung
        \draw [{Stealth[length=5mm]}-, red] (mitte.south) to[bend right=30] node[below, midway] {Rückkopplung} (ende.south);
    }


    \emphColA{\textbackslash{}LoesungTikz\{Lösungstext\}\{Originaltext\}}: 
    % \LoesungTikz{Tikz Diagramm-Code}{Anzahl Kästchen}
    \LoesungKaroTikz{
        % 1. Knoten definieren
        % \node [stil] (interner_name) {Angezeigter Text};
        \node [oval, text width=2cm] (mitte) {Verarbeitung};
        \node [oval, text width=1cm, right=0.5cm of mitte] (ende) {Ende};

        % 2. Knoten mit Pfeilen verbinden
        % \draw [stil] (start_knoten) -- (ziel_knoten);
        \draw [-{Stealth[length=3mm]}] (mitte) -> (ende);
        
        % Ein zusätzlicher, gebogener Pfeil mit Beschriftung
        \draw [{Stealth[length=3mm]}-, red] (mitte.south) to[bend right=30] node[below, midway] {Rückkopplung} (ende.south);
    }{2}

    \emphColA{\textbackslash{}LoesungLine\{Lösungstext\}\{Anzahl Zeilen\}}: 
    % \LoesungLine{Lösungstext}{Anzahl Zeilen}
    \LoesungLine{Erzeugt in der Schülerversion die festgelegte Anzahl an Zeilen zum Schreiben in Seitenbreite.}{1}

    % \LoesungMulti{Lösungstext}{Anzahl Zeilen}
    \emphColA{\textbackslash{}LoesungMulti\{Falsche mit \textbackslash{}multi\{\} und Richtige mit \textbackslash{}Lmulti\{\}\}}: 
    \LoesungMulti{
        \multi{Eine falsche Antwort}
        \Lmulti{Die richtige Antwort}
        \Lmulti{Die andere richtige Antwort}
        \multi{Noch eine falsche Antwort}
    }
}