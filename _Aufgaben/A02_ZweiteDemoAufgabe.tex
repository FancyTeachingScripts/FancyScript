%%%%%%%%%%
% \Aufgabe[Optional: Bearbeitungszeit in Minuten]{Titel}{Inhalt}
%%%%%%%%%%
\Aufgabe{Dateianhang und nochmal Links/QR-Codes}
{
    % \AttachVlgLsg{Symbol}{Vorlagedatei}{Lösungsdatei}
    % \AttachVlg{Symbol}{Vorlagedatei}
    % \AttachLsg{Symbol}{Lösungsdatei}
    \AttachVlgLsg{\faFilePdfO}{_Aufgaben/img/fontawesome.pdf}{sty/img/logo_default.pdf} 

    Wenn für eine Aufgabe Vorlage-Dateien (z.B. Exceltabellen oder Druckvorlagen für eine Stationenarbeit o.ä.) benötigt werden, kann man diese an das PDF anhängen/in es einbetten.
    Genauso auch mit Lösungsdateien (dann z.B. die ausgefüllte Tabelle), die werden natürlich nur im Lösungsmodus angezeigt.

    Zum Öffnen kann man das Symbol bzw. den Text Vorlage/Lsg (doppelt) anklicken. Folgende Optionen stehen zur Verfügung:

    \begin{itemize}
        \item \textbackslash{}AttachVlgLsg\{Symbol\}\{Vorlagedatei\}\{Lösungsdatei\}
        \item \textbackslash{}AttachVlg\{Symbol\}\{Vorlagedatei\}
        \item \textbackslash{}AttachLsg\{Symbol\}\{Lösungsdatei\}
    \end{itemize}

    Eine Übersicht über verfügbare Symbole gibt es in der angehängten Vorlagedatei und hier: \UrlAndCode{satztexnik.com/tex-archive/fonts/fontawesome/doc/fontawesome.pdf}.
    Die Lösungsdatei enthält das FancyTeachingScripts-Logo.    
    
    Achtung: Wenn die UrlAndCode-Links über zwei Zeilen laufen (so wie hier), haben manche PDF Reader Probleme, sie korrekt zu öffnen. Daher besser vermeiden.
    
}




