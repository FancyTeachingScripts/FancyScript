%%%%%%%%%%
% \Aufgabe[Optional: Bearbeitungszeit in Minuten]{Titel}{Inhalt}
%%%%%%%%%%
\Aufgabe[10]{Unteraufgaben, Arbeitszeiten-Timer \& Skript vs. Präsentation}
{
    Ich bin eine Aufgabe mit 10 Minuten Arbeitszeit. In der Druckversion wird sie nicht angezeigt. Nur in den Präsentationen mit aktiviertem Timer.  Dort wird die verbleibende Arbeitszeit dann unten als Balken angezeigt. Pro Minute wird eine PDF-Seite erzeugt, die automatisch nach einer Minute weiterschaltet (+ eine Seite, bevor der Timer läuft, und eine danach). Das geht aber nur mit Acrobat Reader, Foxit Reader und Okular. 
}

\UnterAufgabe[5]{Ich bin eine Unteraufgabe}
{
    Grundsätzlich werden Aufgaben und Hefteinträge immer auf einer Seite zusammengehalten. Wenn ein Hefteintrag nicht auf eine Seite passt, ist er zu lang. Daher lässt sich das hierfür nicht deaktivieren. Bei Aufgaben gibt es die Möglichkeit, eine Aufgabe zu beenden und eine neue Unteraufgabe zu beginnen. 
    
    Der Titel der Unteraufgabe wird nur in der Präsentation angezeigt, möchte man nur im gedruckten Skript auch einen eigenen Text anzeigen, kann man das so tun:

    \ifbeamer
        \emphColC{Dieser Teil wird nur in der Präsentation angezeigt.}
    \else
        \emphColB{Dieser Teil wird nur im gedruckten Skript angezeigt.}
    \fi

    Wenn man einen Block leer lässt, wird im jeweiligen Fall nichts angezeigt. Man kann das nächste Schlüsselwort direkt hinter das vorherige schreiben.
    \ifbeamer\else \emphColA{Sieht dann so aus.} \fi
    Jede Unteraufgabe kann eine eigene Arbeitszeit haben. Diese hier hat 5 Minuten.
}