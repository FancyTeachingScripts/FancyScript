%%%%%%%%%%
% \Aufgabe[Optional: Bearbeitungszeit in Minuten]{Titel}{Inhalt}
%%%%%%%%%%
\Aufgabe{Farben, Links und QR-Codes}
{
    %%%%%
    % Farbige, fette Hervorhebung in den Highlight-Farben des Themes (colA,colB,colC)
    % mit \emphColA{...}, \emphColB{...}, \emphColC{...} oder 
    % freie Farbwahl mit \emphColor{farbe}{...}
    %%%%%
    \emphColA{Ich} \emphColB{bin} \emphColC{eine} \emphColor{red}{Aufgabe} ohne angegebene Arbeitszeit. In der Druckversion würde die aber sowieso nicht angezeigt. Nur in der Präsentation mit aktiviertem Timer.

    Man kann links so einfügen, dass ca. auf Höhe der Überschrift automatisch ein QR-Code erzeugt wird. Wichtig: Link \emph{ohne https://} am Anfang einfügen: \UrlAndCode{www.youtube.com/watch?v=xvFZjo5PgG0}
    
    Link und Code sind im PDF anklickbar. Nur URL ohne QR-Code kann man entweder mit url: \url{https://github.com/FancyTeachingScripts} oder href: \href{https://valentin-herrmann.com/}{Eine Internetseite} einfügen.

}